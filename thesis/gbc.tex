\section{Grammar-Based Codes}
Recently, Kieffer and Yang \cite{Kieffer-00a} \cite{Kieffer-00b} have proposed a new class of lossless source codes caled \concept{grammar-based codes}. In the grammar-based code, a data sequence to be compressed is first converted into a context-free grammar by so called \concept{grammar transform}, and then losslessly encoded.

\subsection{Context-free grammars}

Let $\mathcal{T}$ be the source alphabet of size at least 2. Let $\mathcal{T}^*$ be the set of all finite strings from $\mathcal{T}$ (including the empty string $\lambda$), and $\mathcal{T}^+$ the set of all nonempty strings from $\mathcal{T}$. The cardiniality of $\mathcal{T}$ is denoted as $|\mathcal{T}|$, and for any $x \in \mathcal{T}$, $|x|$ represents the length of $x$.

Let $\mathcal{V} = \{v_0, v_1, v_2, \dots\}$ be a countable set of symbols not appearing in $\mathcal{T}$. This symbols are called \concept{variables}; the symbols in $\mathcal{T}$ are called \concept{terminal symbols}. For any $j \geq 1$, let $\mathcal{V}(j) = \{v_o, v_1, \dots, v_{j-1}\}$. The set $\mathcal{V}(j)$ shall be called the \concept{variable set} of $G$. The variable $v_0$ represents the \concept{start symbol}.

\definition{Context-free grammar}{A \concept{context-free grammar} $G$ is a mapping from $\mathcal{V}(j)$ to $(\mathcal{V}(j) \cup \mathcal{T})^+$ for some $j \geq 1$.}

\definition{Production rule}{The mapping $v_i \rightarrow G(v_i)$ ($0 \leq i < j$) is called a \concept{production rule}.}

The grammar $G$ is completely described by the set of the production rules $\{v_i \rightarrow G(v_i) : 0 \leq i < j\}$.

Let $G$ be an context-free grammar. If $\alpha$ and $\beta$ are strings from $(\mathcal{V}(j) \cup \mathcal{T})^+$, we shall write:

\begin{itemize}
\item $\alpha \stackrel{G}{\rightarrow} \beta$ if there are strings $\alpha_1$, $\alpha_2$ and a production rule $v_i \rightarrow G(v_i)$ such that $(\alpha_1, v_i, \alpha_2)$ is a parsing of $\alpha$ and $(\alpha_1, G(v_i), \alpha_2)$ is a parsing of $\beta$;

\item $\alpha \stackrel{G}{\Rightarrow} \beta$ if there exist a sequence of strings $\alpha_1, \alpha_2, \dots, \alpha_k$ such that $\alpha = \alpha_1 \stackrel{G}{\rightarrow} \alpha_2, \alpha_2 \stackrel{G}{\rightarrow} \alpha_3, \dots, \alpha_{k-1} \stackrel{G}{\rightarrow} \alpha_k = \beta$.
\end{itemize}
\definition{Language generated by the grammar}{The language $L(G)$ \concept{generated by} the grammar $G$ is represented by the set $\{x \in \mathcal{T}^+ : v_0 \stackrel{G}{\Rightarrow} x\}$}

%%%%%%%%%%%%%%%%%%%%%%%%%%%%%%%%%%%%%%%%%

\subsection{Admissible grammars}
\definition{Admissible grammar}{We define a context-free grammar $G$ to be \concept{admissible} if the following properties hold:\footnote{It can be easily seen that by our definition of a context-free grammar, the properties \ref{admissible-deterministic} and \ref{admissible-empty-string} are automatically satisfied.}
\begin{enumerate}
\item \label{admissible-deterministic} $G$ is \concept{determistic} (in other words, for each variable $v_i$ in the variable set of $G$, there is exactly one production rule whose left member is $v_i$);

\item \label{admissible-empty-string} the empty string is not the right member of any production rule in $G$;

\item $L(G)$ is nonempty;
\item $G$ has no useless symbols. This means that for each symbol $x \in \mathcal{V}(j) \cup \mathcal{T}, x \neq v_0$, there exist finitely many strings $\alpha_1, \alpha_2, \dots, \alpha_n$ such that at least one of the strings contains $x$ and $v_0 = \alpha_1 \stackrel{G}{\rightarrow} \alpha_2, \alpha_2 \stackrel{G}{\rightarrow} \alpha_3, \dots, \alpha_{n-1} \stackrel{G}{\rightarrow} \alpha_n \in L(G)$.
\end{enumerate}}

For any deterministic grammar $G$, the language $L(G)$ is either empty or consists of exactly one string. If $G$ is an admissible grammar, there exists an unique string $x \in \mathcal{T}^+$ such that $L(G) = \{x\}$. We shall say that $G$ \concept{represents} $x$, and write $x \rightarrow G_x$.

\definition{The size of the grammar}{Let $G$ be an admissible grammar with variable set $\mathcal{V}(j)$. The size $|G|$ of $G$ is defined as the sum:
$$|G| \stackrel{\Delta}{=} \sum_{v \in \mathcal{V}(j)} |G(v)|$$
where $|G(v)|$ denotes the length of the string $G(v)$.}

\example{Let $\mathcal{T} = \{0, 1\}$. We will present an example of admissible grammar $G$ with the variable set $\{v_0, v_1, v_2, v_3\}$. The grammar represents the sequence $x = 00101101101010101110$ and its size is equal to 14.
\begin{eqnarray*}
v_0 & \rightarrow & 0v_3v_2v_1v_1v_310 \\
v_1 & \rightarrow & 01 \\
v_2 & \rightarrow & v_11 \\
v_3 & \rightarrow & v_1v_2
\end{eqnarray*}}
\label{example-grammar-size}

\remark{Admissible grammars have one important property: We need only list the production rules to fully specify the grammar, because the terminal alphabet, as well as the variable set of the grammar and the start symbol, can be uniquely inferred from the production rules. The set of variables will consist of the left members of the production rules. The terminal alphabet will consist of the symbols which appear in the right members of the production rules and which aren't variables. Finally, the start symbol is the unique variable that doesn't appear in the right members of the production rules.}

\subsection{Irreducible grammars}
It is obvious that for a string $x \in \mathcal{T}^+$, especially when $|x|$ is large, there are many admissible grammars that represent $x$. Some of these grammars can be more compact than others in the sense of having smaller size $|G|$. Consider for example the grammar from the \niceref{example-grammar-size} and a (rather simplistic) grammar with the only one production rule: $v_0 \rightarrow 00101101101010101110$. Both grammars represent the same data string, but the size of the second grammar is 20!


Since the size of $G$ is influential in the performance of the grammar-based code, the grammars should be designed such that the following properties hold:

\begin{propertylist}
\item \label{property-a-1} The size $|G|$ should be small enough, compared to the length of $x$;
\item \label{property-a-2} strings represented by distinct variables of $G$ are distinct;
\item \label{property-a-3} the frequency distribution of variables and terminal symbols of $G$ in the range of $G$ should be such that effective arithmetic coding can be accomplished later on.
\end{propertylist}

Kieffer and Yang have proposed a set of \concept{reduction rules} which, when applied repeatedly to an admissible grammar $G$, lead to another admissible grammar $G'$ which represents the same data string and satisfies the properties \niceref{property-a-1}, \niceref{property-a-2}, and \niceref{property-a-3} in some sense. These reduction rules will be described in the next subsection.

\subsubsection{Reduction Rules}
\simplesection{Reduction Rule 1}{Let $v$ be an variable of an admissible grammar $G$ that appears only once in the range of $G$. Let $v' \rightarrow \alpha v \beta$ be the unique production rule in which $v$ appears on the right. Let $v \rightarrow \gamma$ be the production rule corresponding to $v$. Reduce $G$ to the admissible grammar $G'$ obtained by removing the production rule $v \rightarrow \gamma$ from $G$ and replacing the production rule $v' \rightarrow \alpha v \beta$ with the production rule $s' \rightarrow \alpha \gamma \beta$. The resulting admissible grammar $G'$ represents the same sequence $x$ as does $G$.}

\example{Consider the grammar $G$ with variable set $\{v_0, v_1, v_2\}$ given by $\{v_0 \rightarrow v_1v_1, v_1 \rightarrow v_21, v_2 \rightarrow 010\}$. Applying Reduction Rule 1, one gets the grammar $G'$ with variable set $\{v_0, v_1\}$ given by $\{v_0 \rightarrow v_1v_1, v_1 \rightarrow 0101\}$.}

\simplesection{Reduction Rule 2}{Let $G$ be an admissible grammar possessing a production rule of form $v \rightarrow \alpha_1\beta\alpha_2\beta\alpha_3$, where the length of $\beta$ is at least 2. Let $v' \in \mathcal{V}$ be a variable which is not in $G$. Reduce $G$ to the grammar $G'$ obtained by replacing the production rule $v \rightarrow \alpha_1\beta\alpha_2\beta\alpha_3$ of $G$ with $v \rightarrow \alpha_1v'\alpha_2v'\alpha_3$, and by appending the production rule $v' \rightarrow \beta$. The resulring grammar $G'$ includes a new variable $v'$ and represents the same sequence $x$ as does $G$.}

\example{Consider the grammar $G$ with variable set $\{v_0, v_1\}$ given by $\{v_0 \rightarrow v_101v_101, v_1 \rightarrow 11\}$. Applying Reduction Rule 2, one gets the grammar $G'$ with variable set $\{v_0, v_1, v_2\}$ given by $\{v_0 \rightarrow v_1v_2v_1v_2, v_1 \rightarrow 11, v_2 \rightarrow 01\}$.}

\simplesection{Reduction Rule 3}{Let $G$ be an admissible grammar possessing two distinct production rules of form $v \rightarrow \alpha_1\beta\alpha_2$ and $v' \rightarrow \alpha_3\beta\alpha_4$, where $\beta$ is of length at least 2, either $\alpha_1$ or $\alpha_2$ is not empty, and either $\alpha_3$ or $\alpha_4$ is not empty. Let $v'' \in \mathcal{V}$ be a variable which is not in $G$. Reduce $G$ to the grammar $G'$ obtained by doing the following: Replace rule $v \rightarrow \alpha_1\beta\alpha_2$ by $v \rightarrow \alpha_1v''\alpha_2$, replace rule $v' \rightarrow \alpha_3\beta\alpha_4$ by $v' \rightarrow \alpha_1v''\alpha_2$, and append the new rule $v'' \rightarrow \beta$.}

\example{Consider the grammar $G$ with variable set $\{v_0, v_1, v_2\}$ given by $\{v_0 \rightarrow v_10v_2, v_1 \rightarrow 10, v_2 \rightarrow 0v_10\}$. Applying Reduction Rule 3, one gets the grammar $G'$ with variable set $\{v_0, v_1, v_2, v_3\}$ given by $\{v_0 \rightarrow v_3v_2, v_1 \rightarrow 10, v_2 \rightarrow 0v_3, v_3 \rightarrow 11\}$.}

\simplesection{Reduction Rule 4}{Let $G$ be an admissible grammar possessing two distinct production rules of the form $v \rightarrow \alpha_1\beta\alpha_2$ and $v' \rightarrow \beta$, where $\beta$ is of length at least 2, and either $\alpha_1$ or $\alpha_2$ is not empty. Reduce $G$ to the grammar $G'$ obtained by replacing the production rule $v \rightarrow \alpha_1\beta\alpha_2$ with the production rule $v \rightarrow \alpha_1v'\alpha_2$.}

\example{Consider the grammar $G$ with variable set $\{v_0, v_1, v_2\}$ given by $\{v_0 \rightarrow v_201v_1, v_1 \rightarrow v_20, v_2 \rightarrow 11\}$. Applying Reduction Rule 4, one gets the grammar $G'$ with variable set $\{v_0, v_1, v_2\}$ given by $\{v_0 \rightarrow v_11v_1, v_1 \rightarrow v_20, v_2 \rightarrow 11\}$.}

\simplesection{Reduction Rule 5}{Let $G$ be an admissible grammar in which two variables $v$ and $v'$ represent the same subsequence of the data string represented by $G$. Reduce $G$ to the grammar $G'$ obtained by replacing each appearance of $v'$ in the range of $G$ by $v$ and deleting the production rule corresponding to $v'$. The grammar $G'$ may not be admissible since some further variables of $G'$ may become useless. If so, further reduce $G'$ to the admissible grammar $G''$ obtained by deleting all production rules corresponding to useless variables of $G'$. Both $G$ and $G''$ represent the same data string.}

\remark{It is possible to define more reduction rules than Reduction Rules 1-5. For example, if the right members of the production rules of the admissible grammar $G$ contain non-overlapping substrings $\alpha \neq \alpha'$ representing the same subsequence of the data string represented by $G$, one can reduce $G$ by replacing $\alpha$ and $\alpha'$ with a new variable $v$, while introducing a new production rule (either $v \rightarrow \alpha$ or $v \rightarrow \alpha'$). However, this new rule is somewhat difficult to implement in practice. Kieffer and Yang limited themselves to Reduction Rules 1-5 because they are simple to implement, and yield grammars which are sufficiently reduced.}

\definition{Irreducible grammar}{An admissible grammar $G$ is said to be \concept{irreducible} if none of Reduction Rules 1 to 5 can be applied to $G$ to get a new admissible grammar.}

It follows from the the above definition that each irreducible grammar $G$ satisfies the following properties:

\begin{propertylist}
\item \label{property-b-1} Each variable of $G$ other than $v_0$ (the start symbol) appears at least twice in the range of $G$;
\item \label{property-b-2} there is no non-overlapping repeated pattern of length greater than or equal to 2 in the range of $G$;
\item \label{property-b-3} each distinct variable of $G$ represents a distinct sequence from $\mathcal{T}$.
\end{propertylist}

Property \niceref{property-b-3} holds due to Reduction Rule 5 and is very important to the compression performance of a grammar-based code. A grammar-based code for which the transformed grammar does not satisfy the property \niceref{property-b-3}, may give poor compression results and can not be guaranteed to be universal.


\subsubsection{Grammar transforms}
Let $x$ be a sequence from $\mathcal{T}$ which is to be compressed. A \concept{grammar transform} is a transformation that converts $x$ into an admissible grammar that represents $x$. For our purposes, we are interested particularly in a grammar transform that starts from the grammar $G$ consisting of only one production rule $v_0 \rightarrow x$, and applies repeatedly Reduction Rules 1 to 5 in some order to reduce $G$ into an irreducible grammar $G'$. Such a grammar transform is called an \concept{irreducible} grammar transform. To compress $x$, the corresponding grammar-based code then uses a zero order arithmetic code to compress the irreducible grammar $G'$. 

There are different grammar transforms because of the different orders via which the reduction rules are applied; this results in different grammar-based codes. However, all these grammar-based codes are universal, as proved by Kieffer and Yang in \cite{Kieffer-00a}:

\theorem{Let $G$ be an irreducible grammar representing a sequence $x$ from $\mathcal{T}$. The size $|G|$ of $G$ divided by the length $|x|$ of $x$ goes to $0$ uniformly as $|x|$ increases. Specifically,
\begin{center}
$max\{|G| : G\ is\ irreducible\ grammar\ representing\ x,\ x \in \mathcal{T}^n\} = o(n)$
\end{center}}

\theorem{Any grammar-based code with an irreducible grammar transform is universal in the sense that for any stationary, ergodic source $\{X_i\}^\infty_{i=0}$, the compression rate resulting from using the grammar-based code to compress the first segment $X_1X_2 \dots X_n$ of length $n$ converges, with probability one, to the entropy rate of the source as $n$ goes to infinity.}

The irreducible grammar-based codes combine the power of string matching (Reduction Rules 2 to 4) with that of arithmetic coding, which is the main reason why they are universal.


% In this section, we will show that irreducible grammars satisfy the properties \niceref{property-a-1}, \niceref{property-a-2}, and \niceref{property-a-3} in some sense.

