\chapter{Introduction}
During the short time of its existence, the \concept{Extensible Markup Language} (XML) has become a widely recognized standard for electronic data structuring and exchange.

XML can be used as an data exchange format for \concept{relational database systems} \dots

Being a simple subset of \concept{Standardized General Markup Language} (SGML), it benefits from the parent's strengths, while successfully solving most of the weaknesses. SGML soon turned out to be too generic and complex to implement. Moreover, the potentials of SGML weren't fully utilized very often. Paradoxically, the power of SGML represents its biggest disadvantage. The nature of XML is more restrictive, making therefore many things (such as machine processing) much easier.

XML is a markup language that describes electronic data in the form accessible both to humans and computer programs. A typical XML document is a mixture of text and XML markup which organizes and identifies the individual parts of the document. The markup is represented by \concept{tags} enclosing objects in the data stream. \concept{Start tag} and \concept{end tag}, together with the data between them, form an \concept{element}.

XML is extremely well suited for storing structured and semi-structured texts.
